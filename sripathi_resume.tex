%----------------------------------------------------------------------------------------
%	PACKAGES AND OTHER DOCUMENT CONFIGURATIONS
%----------------------------------------------------------------------------------------

\documentclass{resume} % Use the custom resume.cls style
\usepackage{hyperref} 
\usepackage[left=0.4 in,top=0.4in,right=0.4 in,bottom=0.4in]{geometry} % Document margins
\newcommand{\tab}[1]{\hspace{.2667\textwidth}\rlap{#1}} 
\newcommand{\itab}[1]{\hspace{0em}\rlap{#1}}
\newcommand{\blank}[1]{\hspace*{#1}}
\name{SRIPATHI KAMMARI} % Your name
%% \address{H.NO 1-3-235 TO 236 Chandishwari Nilayam, Kawadiguda, Secuderabad, 500080} % Your address
%% \address{+91 9440739965 \\ kammari.sripathi@gmail.com}% Your phone number and email


\begin{document}
\blank{5 cm}
{Email ID : }
kammari.sripathi@gmail.com \\
\blank{5 cm}
{Contact No : }
+91 9440739965 \\
\blank{5 cm}
{Github Profile : }
\url{https://github.com/ksripathi}
\sectionlineskip \\

%----------------------------------------------------------------------------------------
%	OBJECTIVE
%----------------------------------------------------------------------------------------

\begin{rSection}{CAREER INTERESTS}
  
  { Programming, Automation, Virtualization, Cloud Computing,
    Networks and DevOps on GNU/Linux system based environments }


\end{rSection}

%----------------------------------------------------------------------------------------
%	PROFESSIONAL SUMMARY
%----------------------------------------------------------------------------------------

\begin{rSection}{PROFESSIONAL SUMMARY}
  
  \begin{itemize}
    
  \item 3+ years of experience as a DevOps and System
    Engineering on multiple flavors of GNU/Linux

  \item High ability to work on Agile projects

  \item Hands on experience in developing end-to-end micro
    services using python Flask, Tornado frameworks
    
  \item Experience in developing web applications using MVC
    frameworks (AngularJS)
    
  \item Proficient in using Git Version Control System which
    includes issue tracking, branching and release
    management of projects

  \item Experience in writing automation scripts using
    Python and Shell

  \item Expert in developing applications in Literate model
    using GNU Emacs org-mode, Sphinx
    
  \item Experience in Virtualization technologies
    Vagarnt, OpenVZ, Virtualbox, KVM and Docker

  \item Experience in working with AWS cloud platform
    services EC2 and have knowledge on others

  \item Experience in deployment of python, php based
    application on AWS cloud and other platforms using Auto
    Deployment services, Jenkins and Travis CI

  \item Experience in monitoring servers/services using
    Nagios tool
    
  \item Experience in building, automation of network
    cluster models using Ansible
    
  \item Experience in configuration of LAMP stack,
    Nginx, DNS, DHCP, SSH, Router and other servers

  \item Experience in writing firewall (IPtable) rules for
    Redhat CentOS based servers
    
  \item Experience in LDAP configuration and accounts
    management
    
  \item Experience in working with ELK stack (Elasticsearch,
    Logstash and Kibana) for log analysis
    
  \item Experience in setting up OpenEdx platform, course
    management and customizing APIs.

  \end{itemize}

\end{rSection}

%----------------------------------------------------------------------------------------
%	PROFESSIONAL EXPERIENCE
%----------------------------------------------------------------------------------------

\begin{rSection}{PROFESSIONAL EXPERIENCE}

   \begin{rSubsection}
    {VLEAD, Virtual Labs, IIIT-Hyderabad, India}{Sep 2014 - till date}
    {Project Engineer} {(3+ years)} \hfill \break \textbf{Description
      :} VLEAD, Virtual Labs Engineering and Architecture Division,
    based in IIIT-Hyderabad campus, is one of the several teams
    working on the Virtual Labs project which initiated by the MHRD
    Govt of India. VLEAD is mandate to design and architect the
    implementation framework for Virtual Labs, providing all kinds of
    infrastructure support and services to manage the 100+ lab
    applications. VLEAD team from IIIT Hyderabad has taken the
    responsibility of being the Central Engineering Team in hosting
    the lab applications and providing infrastructure service and
    support to other engineering teams from various top most
    institutes in India (IITs and NITs) and VLEAD is currently
    maintaining cluster of 100+ VMS on different cloud providers (AWS
    EC2) \\

    \break
    
    \textbf{Role and Responsibilities:}
    \begin{itemize}
      
    \item Develop end-to-end micro services/web-applications
      on demand for Virtual Labs project using literate
      programming model (using GNU Emacs org-mode, Sphinx)

    \item Deploy the applications/services on testing, staging and
      production environments (AWS EC2)

    \item Monitor and mainteanance of test, stage and production
      servers (200+ AWS Server VMs)

    \item Release management of Github projects
      
    \item Packaging servers/services using vagrant,
      docker and VirtualBox

    \item Writing Adapters for management of VMs on AWS EC2,
      OpenVZ platforms using Python's BOTO library

    \item Configuration of CI/CD tools
      
    \item Management of DNS servers

    \item Writing backup, restore, migration of databases
      and YAML scripts

    \item Administration of Github organization repositories
      email, LDAP and other accounts

    \item Guiding team members
    \end{itemize}

    \hfill

    \textbf{Contributions : }
    \begin{itemize}

    \item Implemented multiple adapters in Auto Deployment
      Service for Virtual Labs to deploy labs/services on
      various platforms (AWS EC2 cloud, OpenVZ platform and
      others)
      
    \item Developed several end-to-end level web portals
      which are useful to MHRD team, INDIA for visualizing
      the analytics of labs and information such as user
      feedback, user interests and other details of
      workshops
      
    \item Implemented outreach, feedback, lab-data-service
      and analytics micro services using python's Flask
      framework and ELK stack
      
    \item Contributed for building the network architecture
      for hosting of Virtual Labs and automated the same
      using Ansible tool

    \item Configured Google OAuth and Mozilla Persona
      authorization system for micro services
    \end{itemize}
 
  \end{rSubsection}

\end{rSection}


%	INTERNSHIP/TRAINING
%----------------------------------------------------------------------------------------
\begin{rSection}{INTERNSHIP/TRAINING}
  \begin{rSubsection}
  {Spoorthi Communications Pvt Ltd, Hyderabad} {May 2013 - July
    2013}{Media10}{Summer Intern (2 months)} \textbf{Description :}
  Media10 is an organization under Spoorthi Communications Pvt Ltd
  which is mandate to provide computer software support and
  engineering services required to run 10TV news channel
  \end{rSubsection} 

\end{rSection} 


%----------------------------------------------------------------------------------------
%	EDUCATION SECTION
%----------------------------------------------------------------------------------------

\begin{rSection}{Education}
  {\bf Bachelor of Technology in Computer Science and Engineering} \hfill {May 2010 - May 2014}
  \\ 
  AP IIIT (RGUKT - R.K Valley)
  \\
  Rajiv Gandhi University Of Knowledge Technologies,  CGPA: 7.52/10.00  
  
  {\bf PUC (10+2) in M.bi.P.C} \hfill {Sept 2008 - May 2010}
  \\
  AP IIIT (RGUKT - R.K Valley)
  \\
  Rajiv Gandhi University Of Knowledge Technologies,  CGPA: 8.05/10.00  
  
  {\textbf{SSC, Class X}}  \hfill March 2007 - March  2008 \\
  Vivekananadha Vidhyalayam, Gandhari, Kamareddy, Telangana, 80.33\% 

\end{rSection}

\break

%----------------------------------------------------------------------------------------
%	PROJECTS
%----------------------------------------------------------------------------------------

\textbf{PROJECTS}
\sectionlineskip \\ \hrule \textbf {Project No.1} \\ \textbf{Title : }
Virtual Labs College Cloud Platform \\ \textbf{Client : } MHRD Govt of
India \\ \textbf{Role : } Automation of network cluster creation using
ansible YAML scripts, writing documentation using Emacs Org-mode,
Sphinx readtheorg\\ \textbf{Team Size : } 5 \\ \textbf{Technologies
  Used :} Vagrant, VirtualBox, OpenVZ, Ansible, DNS, DHCP, Router and
OpenEdx Platform \\ \textbf{Description :} Virtual Labs College Cloud
Edition is the Portable-Edition for Virtual Labs. It offers the
offline version of Virtual labs, experiments and theoretical
content. This edition addresses the lack of internet access or poor
internet connectivity across different institutes/colleges. It results
in a cost-effective Learning Management System

\bigskip

\textbf {Project No.2} \\
\textbf{Title : } Auto Deployment Service \\
\textbf{Client : } MHRD Govt of India \\
\textbf{Role : } Implementation of Adapter Modules to Manage
VMS of AWS EC2, OpenVZ, Writing python Unit test cases and
configuring Google Oauth authorization \\
\textbf{Team Size : } 3 \\
\textbf{Technologies Used :} Python Flask, Tornado
frameworks, REST API, JSON, Google Oauth, Boto library \\
\textbf{Description :} Auto Deployment Service is a set
of micro services which designed to enable the continuous
deployment of all the Virtual Labs on multiple platform
providers (e.g AWS EC2)



\bigskip

\textbf {Project No.3} \\
\textbf{Title : } Analytics Service \\
\textbf{Client : } MHRD Govt of India \\
\textbf{Role : } To implement all phases in SDLC  \\
\textbf{Team Size : } 1 \\
\textbf{Technologies Used :} Elasticsearch, Logstash, REST API, JSON, python's Flask
framework \\
\textbf{Description :} It is a
micro service which built on Flask python framework using
elements of ELK (Elasticsearch, Logstash and Kibana)
stack. This service uses REST APIs of elasticsearch database
where all the server logs have been pushed from mutliple
servers using Logstash tool. This service implemented
with set of REST APIs which queries the data from
elasticsearch database using REST API calls to satisfy the
use case requirements


\bigskip

\textbf {Project No.4} \\
\textbf{Title : }  Lab Data Service \\
\textbf{Client : } MHRD Govt of India \\
\textbf{Role : } Writing unit test cases, implementation of
database models, building REST APIs and deployment
application on multiple environments \\
\textbf{Team Size : } 3 \\
\textbf{Technologies Used :} python Flask, SQLAlchemy (ORM), JSON and REST API, Python Unittest \\
\textbf{Description :} Lab Data
Service is a micro service which designed to query all
the information about the labs in JSON format using REST
API(s) interface


\bigskip
\textbf {Project No.5} \\ \textbf{Title : } Outreach Web
Portal\\ \textbf{Client : } MHRD Govt of India \\ \textbf{Role : }
Implementation of front-end, back-end and deployment \\ \textbf{Team
  Size : } 3 \\ \textbf{Technologies Used :} AngularJS, Flask, REST
API, JSON, SQLAlchemy (ORM), Python unittest \\ \textbf{Description :}
This project designed to provide the features to conduct the
outreach activities across the country, INDIA and to accommodate all
the workshop related artifacts, analytics at single place. And this
project built to allow multiple login roles \\


%----------------------------------------------------------------------------------------
%	TECHNICAL STRENGTHS SECTION
%----------------------------------------------------------------------------------------

\begin{rSection}
  {Technical Skill Set}

  \begin{tabular}{ @{} >{\bfseries}l @{\hspace{2ex}} l }
    Programming Languages & Python, Shell, C, Java (Core)
    \\ MVC Frameworks & Flask and AngularJS\\ Protocols and
    APIs & JSON, REST \\ Editor tools & GNU Emacs \\ Servers
    & Apache, Nginx, DNS (Bind), Firewall (IPTable) rules
    and DHCP \\ Automation tools & Ansible, make
    \\ Databases & MySQL, NoSQL \\ Data Analytics Stack &
    ELK (Elasticsearch, Logstash and Kibana) \\ Web
    Technologies & HTML, CSS, Javascript, JQuery and PHP
    \\ Version Control Systems & Git (Github, Gitlab and
    Bitbucket) \\ Virtualization & Vagrant, VirtualBox, KVM,
    OpenVZ, Docker \\ Cloud Computing & AWS (EC2, VPC)
    \\ Continuous Integration & Travis CI, Jenkins \\ Other
    tools & Redmine, MediaWiki
    
  \end{tabular}

\end{rSection}

%----------------------------------------------------------------------------------------
\begin{rSection}{Other Activities}
\begin{itemize}
\item Participated in IEEE 2015 conference in Warangal, India
\item Participated for DevCon Hyderabad 2017 conference held
  at Hyderabad, India
\item Given several demos for multiple workshop events held
  at VLEAD, IIIT-H office
\item Volunteer multiple times for By Election in India
\item Participated 15 days industry oriented summer camp
  conducted by Swecha organization
\item Volunteer in 30 days Virtual Labs integration workshop
  held at IIT-Gawhathi
\item Volunteer for Research and Development show case held
  at IIIT-Hyderabad college
 
\end{itemize}
\end{rSection}


\begin{rSection}{PERSONAL PROFILE}
  \textbf {DOB :} 25 July 1993\\
  \textbf {Marital Status :} Single/Unmarried \\
  \textbf {Father's Name : } Sudershan \\
  \textbf {Languages known :} Telugu, English and Hindi \\

\end{rSection}

\end{document}
