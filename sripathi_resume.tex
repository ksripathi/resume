%----------------------------------------------------------------------------------------
%	PACKAGES AND OTHER DOCUMENT CONFIGURATIONS
%----------------------------------------------------------------------------------------
\documentclass{resume} % Use the custom resume.cls style
\usepackage[normalem]{ulem}
\usepackage{hyperref} 
\usepackage{enumitem}
\usepackage[left=0.4 in,top=0.4in,right=0.4 in,bottom=0.4in]{geometry} % Document margins
\newcommand{\tab}[1]{\hspace{.2667\textwidth}\rlap{#1}} 
\newcommand{\itab}[1]{\hspace{0em}\rlap{#1}}
\newcommand{\blank}[1]{\hspace*{#1}}
\name{SRIPATHI KAMMARI} % Your name
%% \address{H.NO 1-3-235 TO 236 Chandishwari Nilayam, Kawadiguda, Secuderabad, 500080} % Your address
%% \address{+91 9440739965 \\ kammari.sripathi@gmail.com}% Your phone number and email
\begin{document}
\blank{5 cm}
{Email ID} \blank{0.7 cm} :
kammari.sripathi@gmail.com \\
\blank{5 cm} 
{Contact No} \blank{0.3 cm} :
+91 9440739965 \\
\blank{5 cm}
{Github Profile : }
\url{https://github.com/ksripathi}
\sectionlineskip \hfill

%----------------------------------------------------------------------------------------
%	OBJECTIVE
%----------------------------------------------------------------------------------------

\begin{rSection}{CAREER INTERESTS}
  
  { Programming, Automation, Micro services, Virtualization,
    Networking, DevOps and System Engineering in GNU/Linux
    system based environments }

\end{rSection}


\begin{rSection}{CERTIFICATIONS COURSES}
  \begin{itemize}
  \item AWS certified Solutions Architect - Associate 2018 
  \item REST API with Flask and Python - 
  \item Python and Django full stack developer bootcamp
  \item Linux Tutorials and Projects
    
  \end{itemize}

\end{rSection}

%----------------------------------------------------------------------------------------
%	PROFESSIONAL SUMMARY
%----------------------------------------------------------------------------------------

\begin{rSection}{PROFESSIONAL SUMMARY}
  
  \begin{itemize}
    
  \item 3+ years of experience as a DevOps and system engineering on
    multiple flavors of GNU/Linux

  \item High ability to work on Agile projects

  \item Full stack development using Python Flask, Tornado, AngularJS,
    JavaScript, CSS and HTML

  \item Automation using Python and Bash (Shell) programming

  \item Expert in programming using Literate Programming model with
    GNU Emacs org-mode and Sphinx tools

  \item Proficient in GIT version control system, issue tracking,
    branching and release management of github projects

  \item Proficient in working with GNU Emacs editor for development
    environment
    
  \item DevOps using Vagarnt, OpenVZ, VirtualBox, KVM,
    Docker and Docker Swarm

  \item Experience in AWS cloud services EC2, VPC and knowledge on
    other services

  \item Experience in working with ELK stack (Elasticsearch, Logstash
    and Kibana) for log analysis

  \item Experience in deployment of Python and PHP based application
    on AWS cloud and in other platforms using Auto Deployment
    service, Jenkins and Travis CI

  \item Monitoring health status and maintenance of
    servers/services (100+ VMs) of production environment
    using Nagios tool
    
  \item Experience in configuration of network cluster (10+ VMs)
    nodes using Ansible
    
  \item Experience in configuration of LAMP stack,
    Nginx, DNS, DHCP, SSH, Router and other servers

  \item Experience in writing firewall (IPTable) rules for
    RHEL CentOS servers
  \item Experience in system administration tasks which
    includes maintenance of Email, DNS, LDAP, Github and
    other accounts
    
  \item Experience in setting up OpenEdx platform, course
    management and customizing APIs.

  \end{itemize}

\end{rSection}

%----------------------------------------------------------------------------------------
%	PROFESSIONAL EXPERIENCE
%----------------------------------------------------------------------------------------

\begin{rSection}{PROFESSIONAL EXPERIENCE}

   \begin{rSubsection}
    {VLEAD, IIIT-Hyderabad, India}{Sep 2014 - till date}
    {Project Engineer} {(3+ years)}

    %% \textbf{Description
    %%   :} VLEAD, Virtual Labs Engineering and Architecture Division,
    %% based in IIIT-Hyderabad campus, is one of the several teams
    %% working on the Virtual Labs project which initiated by the MHRD
    %% Govt of India. VLEAD is mandate to design and architect the
    %% implementation framework for Virtual Labs, providing all kinds of
    %% infrastructure support and services to manage the 100+ lab
    %% applications. VLEAD team from IIIT Hyderabad has taken the
    %% responsibility of being the Central Engineering Team in hosting
    %% the lab applications and providing infrastructure service and
    %% support to other engineering teams from various top most
    %% institutes in India (IITs and NITs) and VLEAD is currently
    %% maintaining cluster of 100+ VMS on different cloud providers (AWS
    %% EC2) \\
    %% \break
    %% \textbf{Roles and Responsibilities :}
    %% \begin{itemize}
      
    %% \item Develop end to end micro services and web applications for
    %%   Virtual Labs project needs using literate programming model
    %%   (using GNU Emacs org-mode, Sphinx)

    %% \item Deploy the applications/services on test, stage and
    %%   production environment (AWS EC2)

    %% \item Release and source code maintenance of Github projects

    %% \item Log analysis of production servers using ELK stack      
      
    %% \item Monitor and maintenance of test, stage and production
    %%   servers (100+ VMs)
      
    %% \item Packaging servers/services using vagrant,
    %%   docker and VirtualBox

    %% \item Configuration of CI and CD tools
      
    %% \item Writing backup, restore and migration of databases scripts

    %% \item Administration of Github organization repositories,
    %%   email, LDAP and DNS records.

    %% \item Giving technical support to the team members
    %% \end{itemize}

    %% \hfill


%----------------------------------------------------------------------------------------
%	PROJECTS
%----------------------------------------------------------------------------------------

\uline {\textbf{Projects Undertaken :}}
\sectionlineskip

\begin{enumerate}[label=\bfseries\arabic*]

\item \textbf{Analytics Service} (MHRD Govt of India)
  \begin{enumerate}
  \item \textbf{Roles and Responsibilities }
    \begin{enumerate}
    \item Provision and configure ELK stack on test, stage and
      production environment
    \item Writing Elasticsearch queries to fetch log analytics in JSON
      format
    \item Setup and configure Logstash client tool on multiple server
      VMs
    \item Writing pattern matches for different log patterns for
      Logstash tool to push the data into elasticsearch database
    \item Implementation of REST API(s) interface on top of
      Elasticsearch service
    \item Security implementation
    \item Service deployment on test, stage and production environments
    \end{enumerate}
  \item \textbf{Technologies Used } Elasticsearch, Logstash, REST API,
    JSON and Python Flask framework
  \item \textbf{Description } \\ \blank{2 cm} It is a micro
    service which built in Flask Python framework using
    elements of ELK (Elasticsearch, Logstash and Kibana)
    stack. This service consumes REST APIs of elasticsearch
    database and exposes set of REST APIs to satisfy the use
    case scenarios
  \end{enumerate}

\item \textbf {Lab Data Service} (MHRD Govt of India)
  \begin{enumerate}
  \item \textbf{Roles and Responsibilities }
    \begin{enumerate}
    \item Implementation of Database models
    \item Data collection from various sources
    \item Writing database migration, backup and restore scripts using
      Python
    \item Implementation of REST API(s) and documentation
      using GNU Emacs org-mode, sphinx readtheorg
    \item Service deployment on multiple environments
    \end{enumerate}
  \item \textbf{Technologies Used } Python Flask, SQLAlchemy
    (ORM), JSON and REST API and Python Unittest
  \item \textbf{Description } \\ \blank{2 cm} Lab Data
    Service is a micro service which designed to perform
    CRUD operations on Labs data using REST API(s) interface
    and JSON as data exchange protocal
  \end{enumerate}

\item \textbf {Virtual Labs College Cloud Platform} (MHRD Govt of India)
  \begin{enumerate}
    
  \item \textbf{Roles and Responsibilities }
    \begin{enumerate}
    \item Configuration of network cluster nodes (10+
      VMs) using Ansible tool
    \item Model implementation using Literate Programming with GNU Emacs
      org-mode, Sphinx readtheorg
    \item Implementation of Firewall (IPTable) rules for nodes in the
      cluster
    \end{enumerate}
    
  \item \textbf{Description } \\ \blank{2 cm} Virtual Labs College
    Cloud Edition is the Portable-Edition for Virtual Labs. It offers
    the Offline version of Virtual Labs, experiments and theoretical
    content. This edition addresses the lack of Internet access or
    poor internet connectivity across different
    institutes/colleges. It results in a cost-effective Learning
    Management System
    
  \item \textbf{Technologies Used } Vagrant, VirtualBox, OpenVZ,
    Ansible, DNS, DHCP, Router and OpenEdx Platform
  \end{enumerate}


\item \textbf {Auto Deployment Service} (MHRD Govt of India)
  \begin{enumerate}
    
  \item \textbf{Roles and Responsibilities}
    \begin{enumerate}
    \item Implementation of Adapter Modules to manage the VMS of AWS
      EC2, OpenVZ
    \item Implementation of Python Unit test cases
    \item Provision and configure the service on test, stage and
      production environments
    \item Package and publishing service in vagrant boxes
    \item Owns the Github source code
    \item Bug fixing, issue tracking and release management of Github
      repository
    \item Configuring Google Oauth authentication to the service
    \end{enumerate}
    
  \item \textbf{Technologies Used } Python Flask, Tornado
    frameworks, REST API, JSON, Google Oauth and Python Boto
    library
  \item \textbf{Description } \\
    \blank{2 cm}Auto Deployment Service is a set of micro services which designed to enable
    the continuous deployment of all the Virtual Labs on multiple platform
    providers (e.g AWS EC2)
  \end{enumerate}


\end{enumerate}
\end{rSubsection}

\end{rSection}


%----------------------------------------------------------------------------------------
%	TECHNICAL STRENGTHS SECTION
%----------------------------------------------------------------------------------------

\begin{rSection}
  {Technical Skills}

  \begin{tabular}{ @{} >{\bfseries}l @{\hspace{2ex}} l }
    Operation Systems & GNU/Linux and Windows \\
    Programming Languages & Python (Flask), BASH Shell, C and Java (Core)
    \\ Web Technologies & AngularJS, DOM, HTML, CSS,
    JavaScript, JQuery and PHP\\ Protocols and
    APIs & JSON, REST \\ Editor tools & GNU Emacs \\ Servers
    & Apache, Nginx, DNS (Bind), Firewall (IPTable) rules
    and DHCP \\ Automation tools & Ansible, Make
    \\ Databases & MySQL, NoSQL \\ Data Analytics Stack &
    ELK (Elasticsearch, Logstash and Kibana) \\
    Version Control Systems & Git (Github, Gitlab and
    Bitbucket) \\ Virtualization & Vagrant, VirtualBox, KVM,
    OpenVZ and Docker \\ Cloud Computing & AWS (EC2, VPC)
    \\ Continuous Integration & Travis CI and Jenkins \\ Other
    tools & Redmine, MediaWiki
    
  \end{tabular}

\end{rSection}


%----------------------------------------------------------------------------------------
%	EDUCATION SECTION
%----------------------------------------------------------------------------------------

\begin{rSection}{Education}
  {\bf Bachelor of Technology in Computer Science and Engineering} \hfill {May 2010 - May 2014}
  \\ 
  IIIT-IDUPULAPAYA
  \\
  Rajiv Gandhi University Of Knowledge Technologies,  CGPA: 7.68/10.00  
  
  {\bf PUC (10+2) in M.Bi.P.C} \hfill {Sept 2008 - May 2010}
  \\
  IIIT-IDUPULAPAYA
  \\
  Rajiv Gandhi University Of Knowledge Technologies,  CGPA: 8.05/10.00  
  
  {\textbf{SSC, Class 10}}  \hfill March 2007 - March  2008 \\
  Vivekananadha Vidhyalayam, Gandhari, Kamareddy, Telangana, 80.33\% 

\end{rSection}


%	INTERNSHIP/TRAINING
%----------------------------------------------------------------------------------------
\begin{rSection}{INTERNSHIP/TRAINING}
  \begin{rSubsection}
  {Spoorthi Communications Pvt Ltd, Hyderabad} {May 2013 - July
    2013}{Media10}{Summer Intern (2 months)}
  \textbf{Project Undertaken:} RSS Feed Reader
  \end{rSubsection} 

\end{rSection} 

%% \begin{rSection}{PERSONAL PROFILE}
%%   \textbf {DOB :} 25 July 1993\\
%%   \textbf {Marital Status :} Single \\
%%   \textbf {Father's Name : } Sudershan \\
%%   \textbf {Languages known :} Telugu, English and Hindi \\
%%   \textbf {Address : } H.NO 1-3-235 TO 236 Chandishwari Nilayam, Kawadiguda, Hyderabad, 500080
%% \end{rSection}

%% \begin{rSection}{SKILLS}
%%   \begin{itemize}
%%   \item Quick learner
%%   \item Excellent debugging skills
%%   \item Individual and team player
%%   \item Dedication towards deliverables at right time
%%   \item Good Professional
%%   \item Good Interpersonal Skills

%%   \end{itemize}

%% \end{rSection}

%% %----------------------------------------------------------------------------------------
%% \begin{rSection}{Other Activities}
%% \begin{itemize}
%% \item Participated in IEEE 2015 conference in Warangal, India
%% \item Participated for DevCon Hyderabad 2017 conference held
%%   at Hyderabad, India
%% \item Given several demos for multiple workshop events held
%%   at VLEAD, IIIT-H office
%% \item Volunteer multiple times for By Election in India
%% \item Participated 15 days industry oriented summer camp
%%   conducted by Swecha organization
%% \item Volunteer in 30 days Virtual Labs integration workshop
%%   held at IIT-Gawhathi
%% \item Volunteer for Research and Development show case held
%%   at IIIT-Hyderabad college
 
%% \end{itemize}
%% \end{rSection}

\end{document}
